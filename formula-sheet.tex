\documentclass[10pt,a4paper,landscape]{article}
\usepackage[utf8]{inputenc}
\usepackage{multicol}
\usepackage{calc}
\usepackage{ifthen}
\usepackage[landscape]{geometry}
\usepackage{amsmath,amsthm,amsfonts,amssymb}
\usepackage{color,graphicx,overpic}
\usepackage{hyperref}

\pdfinfo{
  /Title (formula-sheet.pdf)
  /Creator (TeX)
  /Producer (pdfTeX 1.40.0)
  /Author (Stefan AAhman)
  /Subject (IF1613)
  /Keywords (if1613, sheet, formulas)}

% This sets page margins to .5 inch if using letter paper, and to 1cm
% if using A4 paper. (This probably isn't strictly necessary.)
% If using another size paper, use default 1cm margins.
\ifthenelse{\lengthtest { \paperwidth = 11in}}
    { \geometry{top=.5in,left=.5in,right=.5in,bottom=.5in} }
    {\ifthenelse{ \lengthtest{ \paperwidth = 297mm}}
        {\geometry{top=1cm,left=1cm,right=1cm,bottom=1cm} }
        {\geometry{top=1cm,left=1cm,right=1cm,bottom=1cm} }
    }

% Turn off header and footer
\pagestyle{empty}

% Redefine section commands to use less space
\makeatletter
\newcommand{\uv}[1]{\mathbf {\hat{#1}}}

% Dielectric constant
\newcommand{\dc}{\ensuremath{\varepsilon_0}}

\renewcommand{\section}{\@startsection{section}{1}{0mm}%
                                {-1ex plus -.5ex minus -.2ex}%
                                {0.5ex plus .2ex}%x
                                {\normalfont\large\bfseries}}
\renewcommand{\subsection}{\@startsection{subsection}{2}{0mm}%
                                {-1explus -.5ex minus -.2ex}%
                                {0.5ex plus .2ex}%
                                {\normalfont\normalsize\bfseries}}
\renewcommand{\subsubsection}{\@startsection{subsubsection}{3}{0mm}%
                                {-1ex plus -.5ex minus -.2ex}%
                                {1ex plus .2ex}%
                                {\normalfont\small\bfseries}}
\makeatother

% Define BibTeX command
\def\BibTeX{{\rm B\kern-.05em{\sc i\kern-.025em b}\kern-.08em
    T\kern-.1667em\lower.7ex\hbox{E}\kern-.125emX}}

% Don't print section numbers
\setcounter{secnumdepth}{0}

\setlength{\parindent}{0pt}
\setlength{\parskip}{0pt plus 0.5ex}

% -----------------------------------------------------------------------

\begin{document}
\raggedright
\footnotesize
\begin{multicols}{4}

% multicol parameters
% These lengths are set only within the two main columns
%\setlength{\columnseprule}{0.25pt}
\setlength{\premulticols}{1pt}
\setlength{\postmulticols}{1pt}
\setlength{\multicolsep}{1pt}
\setlength{\columnsep}{2pt}

\begin{center}
  \Large{\underline{IF1613 - Formelsamling}} \\
\end{center}

\section{Allmänt}

Generaliserade integraler: \\
$dQ = \rho d\tau \text{, } \ast = \text{volym } \tau$ \\
$dQ = \rho_s dS \text{, } \ast = \text{yta } S$ \\
$dQ = \rho_l dl \text{, } \ast = \text{kurva } l$ \\

$\uv{x} \cdot \uv{x} = \uv{y} \cdot \uv{y} = \uv{z} \cdot \uv{z} = 1$ \\
$\uv{x} \cdot \uv{y} = \uv{x} \cdot \uv{z} = \uv{y} \cdot \uv{z} = 0$ \\

\subsection{Kartesiskt koordinatsystem}
$d\vec{l} = dx \cdot \uv{x}+dy \cdot \uv{y}+dz \cdot \uv{z}$ \\
$d\vec{S} = |dx \cdot dy| \cdot \uv{z}$ \\
$d\vec{\tau} = |dx \cdot dy \cdot dz|$

\subsection{Cylindriskt koordinatsystem}
$d\vec{l} = dR \cdot \uv{R} + R d\phi \cdot \uv{\phi} + dz \cdot \uv{z}$ \\
$d\vec{l} = R d\phi \cdot \uv{\phi}$ (bottenplatta) \\
$d\vec{S} = |R_0 d\phi dz| \cdot \uv{R}$ \\
$d\vec{S} = |R d\phi dR| \cdot \uv{z}$ (övre cirkelytan) \\
$d\vec{\tau} = |dR \cdot Rd\phi \cdot dz|$

\subsection{Sfäriskt koordinatsystem}
$d\vec{l} = dr \cdot \uv{r} + r d\theta \cdot \uv{\theta} + r \sin{\theta} d\phi \cdot \uv{\phi} $ \\
$d\vec{S} = |r_0 d\theta \cdot r_0 \sin{\theta} \cdot d\phi| \cdot \uv{r}$ \\
$d\vec{\tau} = |r^2 \sin{\theta} \cdot dr \cdot d\theta \cdot d\phi|$

\section{Elektrostatik}

\begin{itemize}
  \item $C$ - Kapacitans [$C/V$]
  \item $\vec{D}$ - elektriska flödestätheten [$C/m^2$]
  \item $\vec{E}$ = elektrisk fältstyrka [$N/C$]
  \item $\vec{F}$ = kraft [$N$]
  \item $I$ = strömstyrka [$A$]
  \item $Q$ = elektrisk laddning [$C$]
\end{itemize}

\subsection{Konstanter}
Elektriska konstanten (permeabiliteten för fria rymden): \\
$\varepsilon_0 = 8,8541878 \cdot 10^{-12}$

\subsection{Coulombs lag}
$Q_0$:s påverkan på $q$:
\begin{equation} \label{eq:force} \tag{ekv 4:1}
\vec{F_q} = \frac{q \cdot Q_0 \cdot \uv{r}}{4 \pi \dc \cdot r^2} [N]
\end{equation}

\begin{equation} \label{eq:electric-field-intensity} \tag{ekv 4:2}
\vec{E} = \lim\limits_{q \to 0}\frac{\vec{F_q}}{q}=\frac{Q_0 \cdot \uv{r}}{4 \pi \dc \cdot r^2} [N/C]
\end{equation}
\begin{equation} \label{eq:electric-field-strength-generalized} \tag{ekv 4:3}
\vec{E} = \int\limits_\ast\frac{dQ \cdot \uv{r}}{4 \pi \dc \cdot r^2}
\end{equation}

\subsection{Guass lag}
\begin{equation} \label{eq:electric-flux-density} \tag{*}
\vec{D} = \frac{Q_0 \cdot \uv{r}}{4 \pi \cdot r^2} [C/m^2]
\end{equation}
\begin{equation} \label{eq:guass-theorem} \tag{ekv 2:1}
\oint\limits_S \vec{D} \cdot d\vec{S} = Q
\end{equation}

\subsection{Potential}
Potential från punktladdning i vakuum:
\begin{equation} \label{eq:potential} \tag{ekv 4:8}
V = \frac{Q}{4 \pi \dc r} [V]
\end{equation}
\begin{equation} \label{eq:potential-generalized} \tag{*}
V = \int\limits_\ast\frac{dQ}{4 \pi \dc r}
\end{equation}

\subsection{Kapacitans}
\begin{equation} \label{eq:capacitance} \tag{ekv 4:11}
C = \frac{Q}{V_0} [C/V]
\end{equation}

Plattkondensator:
\begin{equation} \label{eq:capacitance-plate-capacitor} \tag{ekv 4:12}
C = \frac{Q}{V_0}=\frac{A\dc}{d} [C/V]
\end{equation}

%%%%%%%%%%%%%%%%%%%%%%%%%%%%%%%%%%%%%%%%%%%%%%%%%%%%%%%%%%%%%%%%%%
\section{Ström och strömtäthet}
\begin{itemize}
  \item $\vec{J}$ = strömtäthet [$A/m^2$]
  \item $\vec{J_s}$ = ytströmtäthet [$A/m$]
  \item $\sigma$ = materialkonstant [$(\Omega m)^{-1}$]
  \item $\rho$ = resistivitet ($1/\sigma$) [$\Omega m$]
\end{itemize}

\subsection{Strömtäthet}
\begin{equation} \label{eq:current-density} \tag{ekv 5:1}
\vec{J} = nq\vec{v}
\end{equation}
Strömmen $I$ genom ytan S:
\begin{equation} \label{eq:current-surface} \tag{ekv 5:2}
I = \int\limits_S \vec{J} \cdot d\vec{S}
\end{equation}
Mobiliteten $\mu$:
\begin{equation} \label{eq:current-density-mobility} \tag{ekv 5:3}
v = |\vec{v}| = \mu |\vec{E}| = \mu \vec{E}
\end{equation}
Relationen mellan strömtäthet och elektrisk fältstyrka
\begin{equation} \label{eq:current-density-electric-field-strength-relation} \tag{ekv 5:4}
\vec{J} = \sigma \vec{E}
\end{equation}
Resistans:
\begin{equation} \label{eq:resistance} \tag{*}
R = \frac{l}{S\sigma} = \frac{l\rho}{S}
\end{equation}
\begin{equation} \label{eq:resistance-capacitance} \tag{*}
RC = \varepsilon_0\varepsilon_r / \sigma
\end{equation}

\section{Magnetostatik}
\begin{itemize}
  \item $\vec{B}$ = magnetisk flödestäthet [$T$]
  \item $\vec{M}$ = magnetisk susceptibilitet [$A/m$]
  \item $\vec{H}$ = magnetisk fältstyrka [$A/m$]
  \item $I_{om}$ = fria strömmen [$A$]
  \item $\vec{m}$ = magnetiskt dipolmoment [$Am^2$]
  \item $\Phi$ = magnetiska flödet [$T/m^2$]
\end{itemize}

\subsection{Konstanter}
Magnetiska konstanten (permeabiliteten för fria rymden): \\
$\mu_0 = 4\pi \cdot 10^{-7}$

\subsection{Biot-Savarts lag}
\begin{equation} \label{eq:biot-savart-law-dl} \tag{ekv 6:1}
\vec{B} = \oint\limits_C \frac{\mu_0 I \cdot d\vec{l}  \times \uv{r}}{4 \pi r^2} [T]
\end{equation}

\begin{equation} \label{eq:biot-savart-law-dtau} \tag{ekv 6:3}
\vec{B} = \int\limits_\tau \frac{\mu_0 \cdot \vec{J} \times \uv{r} \cdot d\tau}{4 \pi r^2} [T]
\end{equation}

\begin{equation} \label{eq:biot-savart-law-dS} \tag{ekv 6:4}
\vec{B} = \int\limits_S \frac{\mu_0 \cdot \vec{J_s} \times \uv{r} \cdot dS}{4 \pi r^2} [T]
\end{equation}

Oändligt långa ledare (symmetri):
\begin{equation} \label{eq:biot-savart-law-inf} \tag{ekv 6:5}
\vec{B} = \frac{\mu_0 I}{2 \pi R} \uv{\phi} [T]
\end{equation}

\subsection{Magnetisk fältstyrka}
Cirkulationssatsen:
\begin{equation} \label{eq:circular-loop} \tag{ekv 6:6}
\oint\limits_C H \cdot d\vec{l} = I_{om} [A]
\end{equation}
\[\vec{B} = \mu_0 \cdot \vec{H} \Longrightarrow \oint\limits_C \vec{B} \cdot d\vec{l} = \mu_0 \cdot I_{om}\]
\[I_{om} = NI \text{, där N är antalet varv}\]
\[\text{Approximering ger: } Bl = \mu_0NI \Longrightarrow B = \frac{\mu_0NI}{l}\]
\subsection{Magnetiskt flöde}
\begin{equation} \label{eq:magnetic-flux} \tag{ekv 6:7}
\Phi = \int\limits_S \vec{B} \cdot d\vec{S}
\end{equation}

\section{Induktion och elektromagnetisk spänning}

\begin{itemize}
  \item $\mathcal{E}$ = elektromotorisk spänning (ems) [$V$]
\end{itemize}
\begin{equation} \label{eq:electromotive-force} \tag{ekv 7:1}
\mathcal{E} = \frac{1}{q} \int\limits_C (\vec{F}-\vec{F_s}) \cdot d\vec{l}
\end{equation}

\subsection{Faradays lag}
\begin{equation} \label{eq:faradays-law} \tag{ekv 7:3}
\mathcal{E} = \int\limits_S - \frac{\partial\vec{B}}{\partial t} \cdot d\vec{S} = - \frac{\partial\Phi}{\partial t}
\end{equation}

\subsection{Ömsesidig induktans}
\[\Phi_{12}(t) = M_{12} \cdot i_1(t)\]
\begin{equation} \label{eq:mutual-inductance} \tag{ekv 7:6}
\mathcal{E}_2 = -\frac{\partial\Phi_{12}}{\partial t} = -M_{12} \frac{\partial i_1}{\partial t}
\end{equation}

\subsection{Självinduktans}
\[\Phi = L \cdot i\]
\begin{equation} \label{eq:self-inductance} \tag{ekv 7:7}
\mathcal{E}_2 = -\frac{\partial\Phi}{\partial t} = -L \frac{\partial i}{\partial t}
\end{equation}

\[\Phi_0 = B \cdot S \text{ och } \Phi = N\Phi_0 \Longrightarrow L = \mu_0 N^2 S / l\]

\section{Magnetiska material}
\section{Maxwells fullständiga ekvationer}

% You can even have references
\rule{0.3\linewidth}{0.25pt}
\scriptsize
\bibliographystyle{abstract}
\bibliography{refFile}
\end{multicols}
\end{document}
